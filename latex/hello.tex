\documentclass{article}
\begin{document}
\title{How to Structure a \LaTeX{} Document}
\author{spike \\
  School of Computing, \\
  CQUPT,\\
  \emph{479381142@qq.com}}
\date{2016/12/23}
\maketitle{}
\begin{abstract}
  In this article, I shall discuss some of the fundamental topics in
  producing a structured document.  This document itself does not go into
  much depth, but is instead the output of an example of how to implement
  structure. Its \LaTeX{} source, when in used with my tutorial
  provides all the relevant information.
\end{abstract}

\section{title}
\label{sec:introduction}

  This small document is designed to illustrate how easy it is to create a well structured
  document within \LaTeX\cite{lamport94}.  You should quickly be able to see how the article
  looks very professional, despite the content being far from academic.  Titles, section
  headings, justified text, text formatting etc., is all there, and you would be surprised
 when you see just how little markup was required to get this output.

\subsection{Top Matter}
\label{sec:top-matter}

 The first thing you normally have is a title of the document, as well as
 information about the author and date of publication.  In \LaTeX{} terms,
 this is all generally referred to as \emph{top matter}.

\subsection{Ariticle Information}
\label{sec:article-information}

\begin{itemize}
\item \verb|title{}| --- The title of the article.
\item \verb|\date |--- The date. Use 使用:
  \begin{itemize}
  \item 
  \end{itemize}
\end{itemize}

\end{document}




